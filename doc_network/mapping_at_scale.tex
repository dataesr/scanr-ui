% Options for packages loaded elsewhere
\PassOptionsToPackage{unicode}{hyperref}
\PassOptionsToPackage{hyphens}{url}
%
\documentclass[
]{article}
\usepackage{amsmath,amssymb}
\usepackage{lmodern}
\usepackage{ifxetex,ifluatex}
\ifnum 0\ifxetex 1\fi\ifluatex 1\fi=0 % if pdftex
  \usepackage[T1]{fontenc}
  \usepackage[utf8]{inputenc}
  \usepackage{textcomp} % provide euro and other symbols
\else % if luatex or xetex
  \usepackage{unicode-math}
  \defaultfontfeatures{Scale=MatchLowercase}
  \defaultfontfeatures[\rmfamily]{Ligatures=TeX,Scale=1}
\fi
% Use upquote if available, for straight quotes in verbatim environments
\IfFileExists{upquote.sty}{\usepackage{upquote}}{}
\IfFileExists{microtype.sty}{% use microtype if available
  \usepackage[]{microtype}
  \UseMicrotypeSet[protrusion]{basicmath} % disable protrusion for tt fonts
}{}
\makeatletter
\@ifundefined{KOMAClassName}{% if non-KOMA class
  \IfFileExists{parskip.sty}{%
    \usepackage{parskip}
  }{% else
    \setlength{\parindent}{0pt}
    \setlength{\parskip}{6pt plus 2pt minus 1pt}}
}{% if KOMA class
  \KOMAoptions{parskip=half}}
\makeatother
\usepackage{xcolor}
\IfFileExists{xurl.sty}{\usepackage{xurl}}{} % add URL line breaks if available
\IfFileExists{bookmark.sty}{\usepackage{bookmark}}{\usepackage{hyperref}}
\hypersetup{
  pdftitle={Mapping scientific communities at scale},
  pdfkeywords={scanR, VOSviewer, scientific ccommunity, research
portal, Elasticsearch, network analysis},
  hidelinks,
  pdfcreator={LaTeX via pandoc}}
\urlstyle{same} % disable monospaced font for URLs
\usepackage[left=3cm, right=3cm, top=3cm, bottom=3cm]{geometry}
\setlength{\emergencystretch}{3em} % prevent overfull lines
\providecommand{\tightlist}{%
  \setlength{\itemsep}{0pt}\setlength{\parskip}{0pt}}
\setcounter{secnumdepth}{-\maxdimen} % remove section numbering
\ifluatex
  \usepackage{selnolig}  % disable illegal ligatures
\fi
\newlength{\cslhangindent}
\setlength{\cslhangindent}{1.5em}
\newlength{\csllabelwidth}
\setlength{\csllabelwidth}{3em}
\newenvironment{CSLReferences}[2] % #1 hanging-ident, #2 entry spacing
 {% don't indent paragraphs
  \setlength{\parindent}{0pt}
  % turn on hanging indent if param 1 is 1
  \ifodd #1 \everypar{\setlength{\hangindent}{\cslhangindent}}\ignorespaces\fi
  % set entry spacing
  \ifnum #2 > 0
  \setlength{\parskip}{#2\baselineskip}
  \fi
 }%
 {}
\usepackage{calc}
\newcommand{\CSLBlock}[1]{#1\hfill\break}
\newcommand{\CSLLeftMargin}[1]{\parbox[t]{\csllabelwidth}{#1}}
\newcommand{\CSLRightInline}[1]{\parbox[t]{\linewidth - \csllabelwidth}{#1}\break}
\newcommand{\CSLIndent}[1]{\hspace{\cslhangindent}#1}
% for compatibility with pandoc 2.10
\newenvironment{cslreferences}%
  {\setlength{\parindent}{0pt}%
  \everypar{\setlength{\hangindent}{\cslhangindent}}\ignorespaces}%
  {\par}

\title{Mapping scientific communities at scale}
\usepackage{authblk}
\author[%
  1%
  ]{%
  Victor Barbier%
  %
  %
}
\author[%
  1%
  ]{%
  Eric Jeangirard%
  %
  %
}
\affil[1]{French Ministry of Higher Education and Research, Paris,
France}
\date{January 2025}

\makeatletter
\def\@maketitle{%
  \newpage \null \vskip 2em
  \begin {center}%
    \let \footnote \thanks
         {\LARGE \@title \par}%
         \vskip 1.5em%
                {\large \lineskip .5em%
                  \begin {tabular}[t]{c}%
                    \@author
                  \end {tabular}\par}%
                                                \vskip 1em{\large \@date}%
  \end {center}%
  \par
  \vskip 1.5em}
\makeatother

\begin{document}
\maketitle

\textbf{Keywords}: open access, open science, open data, open source

\hypertarget{motivation}{%
\section{1. Motivation}\label{motivation}}

Analysing and mapping scientific communities provides an insight into
the structure and evolution of academic disciplines. This involves
providing an analytical and visual representation of the relationships
between entities (e.g.~researchers, research laboratories, research
themes), with the aim, in particular, of understanding the networks and
dynamics of scientific collaboration, and identifying collaborative
groups and their influences. From the point of view of decision-makers,
this type of tool is useful for strategic decision-making with a view to
public policy and funding.

These maps are generally deduced from data in bibliographic databases
(open or proprietary), based on co-publication or citation information.
In the case of co-publications, two entities (authors, for example) will
be linked if they have collaborated (co-published) on a piece of
research. These links are then symmetrical. In the case of citation
links, two authors will be linked if one cites the research work of
another, in the list of references. This is a directed link, as one
author may cite another without this being reciprocal. A lot of recent
work uses this second approach, for example by trying to calculate
composite indicators of novelty (or innovation) based on citation links.

The quality and completeness of the bibliographic metadata used are, of
course, essential if we are to produce a relevant map. Today, the
quality of open citation data still needs to be improved, cf (Alperin et
al. 2024). On the other hand, it is possible to obtain quality metadata
on publications (and therefore links to co-publications). For example,
the French Open Science Monitor (BSO) has compiled a corpus of French
publications with good coverage cf (Chaignon and Egret 2022). This
corpus is exposed in the French research portal scanR (Jeangirard 2024).
This is a corpus containing about 4 millions publications in all
disciplines. These publications have been enriched with disambuation
persistent identifier (PID) on authors, affiliations and topics. For
authors and affiliations, French-specific PID have been used (idref for
authors and RNSR for laboratories) because they have the best coverage,
even if not perfect. For topics, wikidata identifiers has been used cf
(Foppiano and Romary 2020). Other enrichments, like software detection
are also present, and thus usable as entities to map.

\hypertarget{current-limits-of-the-scanr-application}{%
\subsection{1.1 Current limits of the scanR
application}\label{current-limits-of-the-scanr-application}}

\hypertarget{network-analysis-limits}{%
\subsection{1.3 Network analysis limits}\label{network-analysis-limits}}

\hypertarget{network-analysis-at-scale}{%
\section{2. Network analysis at scale}\label{network-analysis-at-scale}}

\hypertarget{focusing-on-strongest-interactions}{%
\subsection{Focusing on strongest
interactions}\label{focusing-on-strongest-interactions}}

\hypertarget{elasticsearch-impl}{%
\subsection{Elasticsearch impl}\label{elasticsearch-impl}}

\hypertarget{vosviewer-implem}{%
\subsection{VOSviewer implem}\label{vosviewer-implem}}

\hypertarget{llm-trick}{%
\subsection{LLM trick}\label{llm-trick}}

\hypertarget{making-insightful-maps}{%
\section{3. Making insightful maps}\label{making-insightful-maps}}

\hypertarget{citation-hot-topics}{%
\subsection{Citation / hot topics}\label{citation-hot-topics}}

\hypertarget{user-interaction}{%
\subsection{User interaction}\label{user-interaction}}

\hypertarget{references}{%
\section*{References}\label{references}}
\addcontentsline{toc}{section}{References}

\hypertarget{refs}{}
\begin{cslreferences}
\leavevmode\hypertarget{ref-alperin2024analysissuitabilityopenalexbibliometric}{}%
Alperin, Juan Pablo, Jason Portenoy, Kyle Demes, Vincent Larivière, and
Stefanie Haustein. 2024. ``An Analysis of the Suitability of Openalex
for Bibliometric Analyses.'' \url{https://arxiv.org/abs/2404.17663}.

\leavevmode\hypertarget{ref-10.1162ux2fqss_a_00179}{}%
Chaignon, Lauranne, and Daniel Egret. 2022. ``Identifying Scientific
Publications Countrywide and Measuring Their Open Access: The Case of
the French Open Science Barometer (Bso).'' \emph{Quantitative Science
Studies} 3 (1): 18--36. \url{https://doi.org/10.1162/qss_a_00179}.

\leavevmode\hypertarget{ref-foppiano2020entity}{}%
Foppiano, Luca, and Laurent Romary. 2020. ``Entity-Fishing: A Dariah
Entity Recognition and Disambiguation Service.'' \emph{Journal of the
Japanese Association for Digital Humanities} 5 (1): 22--60.

\leavevmode\hypertarget{ref-jeangirard:hal-04813230}{}%
Jeangirard, Eric. 2024. ``scanR - Explore public data on French research
and innovation.'' In \emph{euroCRIS SMM 2024}. Paris, France: euroCRIS.
\url{https://hal.science/hal-04813230}.
\end{cslreferences}


\end{document}
